\documentclass[A4paper, 12pt, british, reqno]{amsart}

%%% Contents of the preamble
    % Packages ----------------- Line 12
    % General things ----------- Line 72
    % Font definitions --------- Line 87
    % Theorem environments ----- Line 230
    % Tikzcd ------------------- Line 395
    % Author, title, etc ------- Line 416
    % \begin{document} --------- Line 458

%%% Packages

\usepackage{libertine}
\usepackage[libertine]{newtxmath}
% Local font definition; before fontenc, cf. https://tex.stackexchange.com/a/2867

\usepackage[T1]{fontenc}
% This uses 8-bit font encoding (with 256 glyphs) instead of the default 7-bit font encoding (with 128 glyphs). For example, with this option ö is a single glyph in the font, whereas on the 7-bit font encoding the font ö is made by adding an accent to the existing glyph o. A bad consequence of not using this package is that you cannot properly copy-paste such words form the output pdf file. Also, for some reason, funny stuff happens with |, < and > in text.
% Some people suggest to load fontenc before inputenc, most agree that it does not matter.

\usepackage[utf8]{inputenc}
% When you type ä in an editor set up for utf8, the machine stores the character number 228. When TeX reads the file it finds the character number 228 and the macros of inputenc transform this into \"a. Finally fontenc does its thing and transforms this into the command print character 228 (otherwise the two things would be printed separatedly as explained in fontenc).

\usepackage[UKenglish]{babel}
% To manage culturally determined typographical and similar rules, in this case for british english. Some people suggest to load babel after fontenc to avoid warnings, although most agree that it does not matter.

\usepackage{geometry}
\geometry{margin=3cm}

\usepackage{mathtools}
% Loads the amsmath package (\usepackage{amsmath}: miscellaneous improvements such as the commands \DeclareMathOperator and \text). It fixes some quirks it has and adds some useful settings, symbols and environments. It improves the aesthetics as well.

%\usepackage{amssymb}
% Extended symbol collection, e.g. \Cap and \Cup. More importantly: the \mathbb command! It loads the amsfonts package (\usepackage{amsfonts}: fraktur letters, bold Greek letters...), so we do not need to include it in the preamble anymore.

\usepackage{mathrsfs}
% Font package (only supports upper case letters).

\usepackage{enumitem}
% To control the layout of enumerate, itemize and description. It supersedes the enumerate package.

\usepackage{tikz-cd}
% To draw commutative diagrams.

\usepackage{graphicx}
% An extension of the graphics package, with optional arguments for the \includegraphics command.

\usepackage{todonotes}
% To write to do notes use the command \todo.

\usepackage{xcolor}
% To write in colors.

\usepackage{marginnote}
% To write on margins.

\usepackage{manfnt}
% To draw dangerous bent symbol.

\usepackage{float}
% Improved interface for floating objects such as figures and tables, introducing for example the H modifier to force the position of a float in the page or the boxed float. Should be loaded before hyperref.

\usepackage{hyperref}
% To handle cross-referencing and produce hypertext links in the document. It should be loaded last (with few exceptions), because it redefines many LaTeX commands.
% The backref option inserts links on each bibliography entry to the pages in which the citation was used.
%% The hidelinks option removes colors and boxes around links, but the links remain clickable. On firefox the links are even highlighted when the mouse pointer passes over them.
%\renewcommand{\backref}[1]{$\uparrow$~#1}
% Adds an upwards arrow before referencing to the pages in which the citations appear.

\usepackage[noabbrev]{cleveref}
% Enhances cross-referencing features, e.g. to reference to a theorem and automatically include the word theorem.
% No abbreviature option to write figure instead of fig. etc.

%%% General things

% Custom colors
\definecolor{darkgreen}{RGB}{0,100,0}
\definecolor{darkblue}{RGB}{0,0,100}
\definecolor{darkred}{RGB}{100,0,0}
\definecolor{linkred}{rgb}{0.7,0.2,0.2}
\definecolor{linkblue}{rgb}{0,0.2,0.6}

% Limit table of contents to section titles
\setcounter{tocdepth}{1}

% Sloppy formatting -- often looks better
\sloppy

%%% Font definitions

% Script Font used for sheaves
\DeclareFontFamily{OMS}{rsfs}{\skewchar\font'60}
\DeclareFontShape{OMS}{rsfs}{m}{n}{<-5>rsfs5 <5-7>rsfs7 <7->rsfs10 }{}
\DeclareSymbolFont{rsfs}{OMS}{rsfs}{m}{n}
\DeclareSymbolFontAlphabet{\scr}{rsfs}
\DeclareSymbolFontAlphabet{\scr}{rsfs}

% Sheaves
\newcommand{\sA}{\scr{A}}
\newcommand{\sB}{\scr{B}}
\newcommand{\sC}{\scr{C}}
\newcommand{\sD}{\scr{D}}
\newcommand{\E}{\scr{E}} % Exception (Vector bundles)
\newcommand{\F}{\scr{F}} % Exception (Coherent sheaves)
\newcommand{\G}{\scr{G}} % Exception (Coherent sheaves)
\newcommand{\sH}{\scr{H}}
\renewcommand{\hom}{\scr{H}\negthinspace om} % Exception (Hom-sheaf)
\newcommand{\I}{\scr{I}} % Exception (Ideal sheaves)
\newcommand{\sJ}{\scr{J}}
\newcommand{\sK}{\scr{K}}
\renewcommand{\L}{\scr{L}} % Exception (Line bundles)
\newcommand{\M}{\scr{M}} % Exception (Line bundles)
\newcommand{\sN}{\scr{N}}
\renewcommand{\O}{\scr{O}} % Exception (Structure sheaf)
\newcommand{\sP}{\scr{P}}
\newcommand{\sQ}{\scr{Q}}
\newcommand{\sR}{\scr{R}}
\newcommand{\sS}{\scr{S}}
\newcommand{\sT}{\scr{T}}
\newcommand{\sU}{\scr{U}}
\newcommand{\sV}{\scr{V}}
\newcommand{\sW}{\scr{W}}
\newcommand{\w}{\omega} % Addition (Canonical sheaf)
\newcommand{\sX}{\scr{X}}
\newcommand{\sY}{\scr{Y}}
\newcommand{\sZ}{\scr{Z}}

% Mathcal fonts
\newcommand{\calA}{\mathcal{A}}
\newcommand{\calB}{\mathcal{B}}
\newcommand{\calC}{\mathcal{C}}
\newcommand{\calD}{\mathcal{D}}
\newcommand{\calE}{\mathcal{E}}
\newcommand{\calF}{\mathcal{F}}
\newcommand{\calG}{\mathcal{G}}
\newcommand{\calH}{\mathcal{H}}
\newcommand{\calI}{\mathcal{I}}
\newcommand{\calJ}{\mathcal{J}}
\newcommand{\calK}{\mathcal{K}}
\newcommand{\calL}{\mathcal{L}}
\newcommand{\calM}{\mathcal{M}}
\newcommand{\calN}{\mathcal{N}}
\newcommand{\calO}{\mathcal{O}}
\newcommand{\calP}{\mathcal{P}}
\newcommand{\calQ}{\mathcal{Q}}
\newcommand{\calR}{\mathcal{R}}
\newcommand{\calS}{\mathcal{S}}
\newcommand{\calT}{\mathcal{T}}
\newcommand{\U}{\mathcal{U}} % Exception (Open covers)
\newcommand{\calV}{\mathcal{V}}
\newcommand{\calW}{\mathcal{W}}
\newcommand{\X}{\mathcal{X}} % Exception (Families of varieties)
\newcommand{\Y}{\mathcal{Y}} % Exception (Families of varieties)
\newcommand{\calZ}{\mathcal{Z}}

% Blackboard Bold Symbols
\newcommand{\A}{\mathbb{A}} % Exception (Affine space)
\newcommand{\bbB}{\mathbb{B}}
\newcommand{\C}{\mathbb{C}} % Exception (Complex numbers)
\newcommand{\bbD}{\mathbb{D}}
\newcommand{\bbE}{\mathbb{E}}
\newcommand{\bbF}{\mathbb{F}}
\newcommand{\bbG}{\mathbb{G}}
\newcommand{\Gm}{\mathbb{G}_{\mathrm{m}}} % Addition (Punctured affine line)
\newcommand{\bbH}{\mathbb{H}}
\newcommand{\bbI}{\mathbb{I}}
\newcommand{\bbJ}{\mathbb{J}}
\newcommand{\bbK}{\mathbb{K}}
\newcommand{\bbL}{\mathbb{L}}
\newcommand{\bbM}{\mathbb{M}}
\newcommand{\N}{\mathbb{N}} % Exception (Natural numbers)
\newcommand{\bbO}{\mathbb{O}}
\renewcommand{\P}{\mathbb{P}} % Exception (Projective space)
\newcommand{\Q}{\mathbb{Q}} % Exception (Rational numbers)
\newcommand{\R}{\mathbb{R}} % Exception (Real numbers)
\newcommand{\bbS}{\mathbb{S}}
\newcommand{\bbT}{\mathbb{T}}
\newcommand{\bbU}{\mathbb{U}}
\newcommand{\V}{\mathbb{V}} % Exception (Geometric vector bundle)
\newcommand{\bbW}{\mathbb{W}}
\newcommand{\bbX}{\mathbb{X}}
\newcommand{\bbY}{\mathbb{Y}}
\newcommand{\Z}{\mathbb{Z}} % Exception (Integers)

% Boldfont (categories)
\newcommand{\bfA}{\mathbf{A}}
\newcommand{\Ab}{\mathbf{Ab}}
\newcommand{\bfB}{\mathbf{B}}
\newcommand{\bfC}{\mathbf{C}}
\newcommand{\Cat}{\mathbf{Cat}} % Addition (Categories)
\newcommand{\Coh}{\mathbf{Coh}} % Addition (Coherent sheaves)
\newcommand{\D}{\mathbf{D}} % Exception (Derived category)
\newcommand{\Db}{\mathbf{D}^{\mathrm{b}}} % Addition (Bounded derived category)
\newcommand{\bfE}{\mathbf{E}}
\newcommand{\bfF}{\mathbf{F}}
\newcommand{\bfG}{\mathbf{G}}
\newcommand{\bfH}{\mathbf{H}}
\newcommand{\bfI}{\mathbf{I}}
\newcommand{\bfJ}{\mathbf{J}}
\newcommand{\K}{\mathbf{K}} % Exception (Homotopy category)
\newcommand{\bfL}{\mathbf{L}}
\newcommand{\bfM}{\mathbf{M}}
\newcommand{\Mod}{\mathbf{Mod}} % Addition (Modules)
\newcommand{\bfN}{\mathbf{N}}
\newcommand{\bfO}{\mathbf{O}}
\newcommand{\bfP}{\mathbf{P}}
\newcommand{\PSh}{\mathbf{PSh}} % Addition (Presheaves)
\newcommand{\bfQ}{\mathbf{Q}}
\newcommand{\QCoh}{\mathbf{QCoh}} % Addition (Quasi-coherent sheaves)
\newcommand{\bfR}{\mathbf{R}}
\newcommand{\bfS}{\mathbf{S}}
\newcommand{\Set}{\mathbf{Set}} % Addition (Sets)
\newcommand{\Sh}{\mathbf{Sh}} % Addition (Sheaves)
\newcommand{\bfT}{\mathbf{T}}
\newcommand{\bfU}{\mathbf{U}}
\newcommand{\bfV}{\mathbf{V}}
\renewcommand{\Vec}{\mathbf{Vec}} % Addition (Vector bundles)
\newcommand{\bfW}{\mathbf{W}}
\newcommand{\bfX}{\mathbf{X}}
\newcommand{\bfY}{\mathbf{Y}}
\newcommand{\bfZ}{\mathbf{Z}}

% Mathfrak for ideals
\renewcommand{\a}{\mathfrak{a}}
\renewcommand{\b}{\mathfrak{b}}
\renewcommand{\c}{\mathfrak{c}}
\renewcommand{\d}{\mathfrak{d}}
\newcommand{\e}{\mathfrak{e}}
\newcommand{\m}{\mathfrak{m}}
\newcommand{\n}{\mathfrak{n}}

%%% Theorem environments

% Custom theorem styles (empty fields take default values)
\newtheoremstyle{darkgreentheorem}% name of the style
{}% measure of space to leave above the theorem. E.g.: 3pt
{}% measure of space to leave below the theorem. E.g.: 3pt
{\itshape}% name of font to use in the body of the theorem
{}% measure of space to indent
{\color{darkgreen}\bfseries}% name of head font
{.}% punctuation between head and body
{ }% space after theorem head; " " = normal interword space
{}% Manually specify head
\newtheoremstyle{darkbluedefinition}
{}{}{}{}{\color{darkblue}\bfseries}{.}{ }{}
\newtheoremstyle{darkredexample}
{}{}{}{}{\color{darkred}\bfseries}{.}{ }{}

% Numbered theorems
\theoremstyle{plain}
% \theoremstyle{darkgreentheorem}
\newtheorem{thm}{Theorem}[section]
\newtheorem{lm}[thm]{Lemma}
\newtheorem{prop}[thm]{Proposition}
\newtheorem{cor}[thm]{Corollary}
\newtheorem{conj}[thm]{Conjecture}
\newtheorem{fact}[thm]{Fact}
\theoremstyle{definition}
% \theoremstyle{darkbluedefinition}
\newtheorem{defn}[thm]{Definition}
% \theoremstyle{darkredexample}
\newtheorem{exa}[thm]{Example}
\theoremstyle{remark}
\newtheorem{rem}[thm]{Remark}
\newtheorem{nota}[thm]{Notation}
\newtheorem{q}[thm]{Question}
\newtheorem{exe}[thm]{Exercise}

% Custom numbered theorems
\theoremstyle{plain}
% \theoremstyle{darkgreentheorem}
\newtheorem{innercustomthm}{Theorem}
\newenvironment{cthm}[1]
    {\renewcommand\theinnercustomthm{#1}\innercustomthm}
    {\endinnercustomthm}
\newtheorem{innercustomlm}{Lemma}
\newenvironment{clm}[1]
    {\renewcommand\theinnercustomlm{#1}\innercustomlm}
    {\endinnercustomlm}
\newtheorem{innercustomprop}{Proposition}
\newenvironment{cprop}[1]
    {\renewcommand\theinnercustomprop{#1}\innercustomprop}
    {\endinnercustomprop}
\newtheorem{innercustomcor}{Corollary}
\newenvironment{ccor}[1]
    {\renewcommand\theinnercustomcor{#1}\innercustomcor}
    {\endinnercustomcor}
\newtheorem{innercustomconj}{Conjecture}
\newenvironment{cconj}[1]
    {\renewcommand\theinnercustomconj{#1}\innercustomconj}
    {\endinnercustomconj}
\newtheorem{innercustomfact}{Fact}
\newenvironment{cfact}[1]
    {\renewcommand\theinnercustomfact{#1}\innercustomfact}
    {\endinnercustomfact}
% Definitions
\theoremstyle{definition}
% \theoremstyle{darkbluedefinition}
\newtheorem{innercustomdefn}{Definition}
\newenvironment{cdefn}[1]
    {\renewcommand\theinnercustomdefn{#1}\innercustomdefn}
    {\endinnercustomdefn}
% \theoremstyle{darkredexample}
\newtheorem{innercustomexa}{Example}
\newenvironment{cexa}[1]
    {\renewcommand\theinnercustomexa{#1}\innercustomexa}
    {\endinnercustomexa}
\theoremstyle{remark}
\newtheorem{innercustomrem}{Remark}
\newenvironment{crem}[1]
    {\renewcommand\theinnercustomrem{#1}\innercustomrem}
    {\endinnercustomrem}
\newtheorem{innercustomnota}{Notation}
\newenvironment{cnota}[1]
    {\renewcommand\theinnercustomnota{#1}\innercustomnota}
    {\endinnercustomnota}
\newtheorem{innercustomq}{Question}
\newenvironment{cq}[1]
    {\renewcommand\theinnercustomq{#1}\innercustomq}
    {\endinnercustomq}
\newtheorem{innercustomexe}{Exercise}
\newenvironment{cexe}[1]
    {\renewcommand\theinnercustomexe{#1}\innercustomexe}
    {\endinnercustomexe}

% Unnumbered theorems
\theoremstyle{plain}
% \theoremstyle{darkgreentheorem}
\newtheorem*{uthm}{Theorem}
\newtheorem*{ulm}{Lemma}
\newtheorem*{uprop}{Proposition}
\newtheorem*{ucor}{Corollary}
\newtheorem*{uconj}{Conjecture}
\newtheorem*{ufact}{Fact}
\theoremstyle{definition}
% \theoremstyle{darkbluedefinition}
\newtheorem*{udefn}{Definition}
% \theoremstyle{darkredexample}
\newtheorem*{uexa}{Example}
\theoremstyle{remark}
\newtheorem*{urem}{Remark}
\newtheorem*{unota}{Notation}
\newtheorem*{uq}{Question}
\newtheorem*{uexe}{Exercise}

% Cross-referencing
\crefname{thm}{theorem}{theorems}
\Crefname{thm}{Theorem}{Theorems}
\crefname{lm}{lemma}{lemmas}
\Crefname{lm}{Lemma}{Lemmas}
\crefname{prop}{proposition}{propositions}
\Crefname{prop}{Proposition}{Propositions}
\crefname{cor}{corollary}{corollaries}
\Crefname{cor}{Corollary}{Corollaries}
\crefname{conj}{conjecture}{conjectures}
\Crefname{conj}{Conjecture}{Conjectures}
\crefname{fact}{fact}{facts}
\Crefname{fact}{Fact}{Facts}
\crefname{defn}{definition}{definitions}
\Crefname{defn}{Definition}{Definitions}
\crefname{exa}{example}{examples}
\Crefname{exa}{Example}{Examples}
\crefname{rem}{remark}{remarks}
\Crefname{rem}{Remark}{Remarks}
\crefname{nota}{notation}{notations}
\Crefname{nota}{Notation}{Notations}
\crefname{q}{question}{questions}
\Crefname{q}{Question}{Questions}
\crefname{exe}{exercise}{exercises}
\Crefname{exe}{Exercise}{Exercises}
% More cross-referencing
\crefname{cthm}{theorem}{theorems}
\Crefname{cthm}{Theorem}{Theorems}
\crefname{clm}{lemma}{lemmas}
\Crefname{clm}{Lemma}{Lemmas}
\crefname{cprop}{proposition}{propositions}
\Crefname{cprop}{Proposition}{Propositions}
\crefname{ccor}{corollary}{corollaries}
\Crefname{ccor}{Corollary}{Corollaries}
\crefname{cconj}{conjecture}{conjectures}
\Crefname{cconj}{Conjecture}{Conjectures}
\crefname{cfact}{fact}{facts}
\Crefname{cfact}{Fact}{Facts}
\crefname{cdefn}{definition}{definitions}
\Crefname{cdefn}{Definition}{Definitions}
\crefname{cexa}{example}{examples}
\Crefname{cexa}{Example}{Examples}
\crefname{crem}{remark}{remarks}
\Crefname{crem}{Remark}{Remarks}
\crefname{cnota}{notation}{notations}
\Crefname{cnota}{Notation}{Notations}
\crefname{cq}{question}{questions}
\Crefname{cq}{Question}{Questions}
\crefname{cexe}{exercise}{exercises}
\Crefname{cexe}{Exercise}{Exercises}

%%% Tikzcd

% Open and closed immersion arrows.
\makeatletter
\tikzcdset{
open/.code={\tikzcdset{hook, circled};},
closed/.code={\tikzcdset{hook, slashed};},
circled/.code={\tikzcdset{markwith={\draw (0,0) circle (.375ex);}};},
slashed/.code={\tikzcdset{markwith={\draw[-] (-.4ex,-.4ex) -- (.4ex,.4ex);}};},
markwith/.code={
\pgfutil@ifundefined{tikz@library@decorations.markings@loaded}%
{\pgfutil@packageerror{tikz-cd}{You need to say %
\string\usetikzlibrary{decorations.markings} to use arrow with markings}{}}{}%
\pgfkeysalso{/tikz/postaction={/tikz/decorate,
/tikz/decoration={
markings,
mark = at position 0.5 with
{#1}}}}},
}
\makeatother

%%% Author, title, etc.

% Author info
\author{GK1821 ``Cohomological Methods in Geometry''}
\address{Pedro N\'{u}\~{n}ez \newline
\indent Albert-Ludwigs-Universit\"{a}t Freiburg, Mathematisches Institut \newline
\indent Ernst-Zermelo-Straße 1, 79104 Freiburg im Breisgau (Germany)}
\email{\href{mailto:pedro.nunez@math.uni-freiburg.de}{pedro.nunez@math.uni-freiburg.de}}
%\renewcommand*{\urladdrname}{\itshape Homepage}
%\urladdr{\href{https://home.mathematik.uni-freiburg.de/nunez/}{https://home.mathematik.uni-freiburg.de/nunez}}
%\thanks{The author gratefully acknowledges support by the DFG-Graduiertenkolleg GK1821 ``Cohomological Methods in Geometry'' at the University of Freiburg.}

% Content details
%\keywords{...}
%\subjclass[...]{...}
\title[Enumerative Geometry Wednesday Seminar]{Enumerative Geometry Wednesday Seminar}
\date{Winter Semester 2020/2021}

% Links and pdf options
\makeatletter
\hypersetup{
  pdfauthor={\authors},
  pdftitle={\@title},
  %pdfsubject={\@subjclass},
  %pdfkeywords={\@keywords},
  pdfstartview={Fit},
  pdfpagelayout={TwoColumnRight},
  pdfpagemode={UseOutlines},
  bookmarks,
  colorlinks,
  linkcolor=darkblue,
  citecolor=darkgreen,
  urlcolor=darkred}
\makeatother

% Change name of ToC
\addto\captionsUKenglish{
    \renewcommand{\contentsname}
	{List of Talks (assuming 1h 30min / Talk)}
}

% Math operators
\DeclareMathOperator{\Hom}{Hom}
\DeclareMathOperator{\GL}{GL}
\DeclareMathOperator{\Cl}{Cl}
\DeclareMathOperator{\Rat}{Rat}

% Other commands
\newcommand{\ot}{\otimes}
\newcommand{\op}{\oplus}

\begin{document}

%%% Contents of the document
    % Talk 1 ---------------- Line 468
    % Talk 2 ---------------- ?
    % Talk 3 ---------------- ?
    % Talk 4 ---------------- ?
    % Talk 5 ---------------- ?

\maketitle

\vspace{-3mm}

\tableofcontents

\vspace{-9mm}

Some suggestions that apply to all the talks:
\begin{itemize}
    \item Working only over $\C$ sounds like a good idea, even if it is often unnecessary.
    \item The book \textit{3264 and all that} written by Eisenbud and Harris \cite{eh16} should provide a self-contained reference for the seminar.
	They often use schemes, but I think the book has a very geometric flavour and most arguments can be followed without knowing precisely what schemes are.
	There is also an introduction to schemes again with a very geometric flavour by the same authors \cite{eh00}.
    \item In fact I would try to avoid scheme-theoretic details altogether and instead try to draw many pictures and focus on examples with complex manifolds.
    \item There are connections and similarities with other areas, especially with singular homology.
	I think it would be nice to hear about them every now and then.
\end{itemize}

But these are only suggestions, feel free to do otherwise if you want/need at some point!

\section{Talk 1 --- The Chow ring. Affine spaces.}

\subsection{Algebraic varieties}

These have already appeared often in past Wednesday seminars, so hopefully we can keep this to a very brief introduction or recollection.
Possible things that may be useful to say:
\begin{itemize}
    \item Huge open subsets (no strict Hausdorffness, no local $\A^{n}$-ness).
    \item The union of two intersecting lines is connected but not irreducible.
    \item Intersection of irreducible stuff is not necessarily irreducible.
    \item Dimension of varieties is defined by chains of irreducibles.
\end{itemize}

\subsection{Chow groups \cite[\S 1.2.1 and 1.2.2]{eh16}}
Cycles and rational equivalence.
\cite[Prop.~1.4]{eh16} and \cite[Prop.~1.10]{eh16} are good to know.
A nice picture to see what can happen is \cite[Fig.~1.2]{eh16}.

\subsection{Ring structure \cite[Thm.~1.5]{eh16}}
Generic transversality and moving lemma \cite[Thm.~1.6]{eh16}.
What goes wrong without the smoothness assumption?
Example in \cite[p.~20]{eh16}.

\subsection{Chow groups of affine spaces}
For any variety $X$, the equivalence class $[X]$ is a free generator of $A^{0}(X)$.
This can be argued using irreducibility and dimension.
In the case of affine spaces, this free cyclic group is all there is \cite[Prop.~1.13]{eh16}.
A nice picture to visualise the proof is \cite[Fig.~1.7]{eh16}.

\subsection{Functoriality \cite[\S 1.3.6]{eh16}}
Proper pushforward and flat pullback without technical details.
I wouldn't define properness and flatness too seriously, but it is good to know that inclusions of open subsets are flat morphisms and that any morphism between projective varieties is proper.
Degree map \cite[Prop.~1.21]{eh16}.
\cite[Thm.~1.23]{eh16} without details of the proof.

\section{Talk 2 --- Affine stratifications. Projective spaces.}

\subsection{Brief recollection of previous talk} Cycles, rational equivalence, functoriality.

\subsection{Mayer--Vietoris and excision \cite[\S 1.3.4]{eh16}}

\subsection{Affine stratifications \cite[\S 1.3.5]{eh16}}
With examples of what is or isn't a stratification, examples of quasi-affine stratifications that are not affine, etc.
Totaro's theorem \cite[Thm.~1.18]{eh16} is nice to know, although it won't be used later on.

\subsection{Chow ring of projective space} \cite[Thm.~2.1]{eh16} and corollaries \cite[Cor.~2.2 and Cor.~2.3]{eh16}.
Bézout's theorem as a consequence of \cite[Thm.~2.1]{eh16}.

\section{Talk 3 --- Grassmannian of lines in space}

\subsection{Kleiman's transversality \cite[Thm.~1.7]{eh16}}
Proof in the case of $\GL_{n}(\C)$.

\subsection{Grassmannians}
Definition and projective spaces as a particular case.
Pl\"{u}cker embedding and affine open cover already in the case of the Grassmannian of lines in $\P^{3}$.
Schubert cycles and stratification of $\bbG(1,3)$.

\subsection{Computation of the Chow ring \cite[Thm.~3.10]{eh16}}
Mentioning explicitly how to use Kleiman's transversality and the method of undetermined coefficients during the proof.

As an immediate consequence of the previous point and transversality: how many lines in $\P^{3}$ intersect $4$ general lines?

\section{Talk 4 --- Specialisation and Knutson--Tao puzzles}

\subsection{Brief recollection of previous talk}
Using the method of undetermined coefficients to compute the square $\sigma_{1}^{2}$ of the Schubert class of lines intersecting a given line.

\subsection{(Static) specialisation \cite[\S 3.5.1]{eh16}}
This is another useful technique to compute products of Schubert classes.
As an example we can use it to compute $\sigma_{1}^{2}$ in a different way.
This will require describing the tangent space of the Schubert cycle of lines intersecting a given line $L$ at a point different from $L$ \cite[Exe.~3.26]{eh16}.

\subsection{Knutson--Tao puzzles}
These give a nice visual tool to compute products of Schubert classes.
Besides various articles by Knutson, Tao and others, see e.g.~\cite{ktw04}, there is also a nice YouTube video discussing them \href{https://youtu.be/U8sq3BplCfI}{https://youtu.be/U8sq3BplCfI}.
Again, one can use the computation of $\sigma_{1}^{2}$ as an example, as is done in the video already.

The speaker may want to discuss other combinatoric methods to compute products of Schubert classes as well/instead, e.g.~Young diagrams \cite[\S 4.5]{eh16}.
Whatever they prefer.

\section{Talk 5 --- Chern classes and lines on a cubic surface}

\subsection{First Chern class of a line bundle \cite[\S 1.4]{eh16}}
Examples could include $c_{1}(K_{\P^{n}})$.
Geometric motivation to generalise and define higher Chern classes \cite[\S 5.2]{eh16}.

\subsection{Axiomatic definition of Chern classes}
Maybe mentioning some ideas in the existence proof but without getting into details \cite[\S 5.3]{eh16}.

\subsection{The splitting principle \cite[\S 5.4]{eh16}}
And how to use it to derive various formulas \cite[\S 5.5]{eh16}.

\subsection{Tautological bundles \cite[\S 5.6]{eh16}}
And computation of the Chern classes of the dual of the tautological subbundle of rank $2$ on our Grassmannian $\bbG(1,3)$ as an example \cite[\S 5.6.2]{eh16}.

\subsection{Counting lines on a cubic surface}
Following the argument in \cite[Thm.~5.1]{eh16} and using \cite[\S 5.6.2]{eh16} and \cite[\S 6.2.1]{eh16}.
Emphasis on what the argument in \cite[Thm.~5.1]{eh16} does and does not show, i.e.~there are some statements that remain to be shown in order to actually prove that there are exactly 27 lines.
May be useful to compare this with the discussion in \cite[\S 3.1]{eh16}.
If time permits and the speaker wants, maybe also briefly sketch the rest of the argument involving Fano varieties \cite[\S 6.1--6.2]{eh16}.

\bibliographystyle{alpha}
\bibliography{main}
\vfill

\end{document}
